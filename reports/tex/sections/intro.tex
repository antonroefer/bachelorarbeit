\section{Einführung}

\subsection{Ground Penetrating Radar}

Das Ground Penetrating Radar (GPR), auch Bodenradar genannt, ist ein geophysikalisches Messverfahren zur Untersuchung des Untergrunds mittels elektromagnetischer Wellen im Mega- bis Gigahertz-Bereich. Ein typisches GPR-System besteht aus einer Sendeantenne, die kurze elektromagnetische Impulse in den Boden abstrahlt, und einer Empfängerantenne, die die vom Untergrund reflektierten Signale aufnimmt. Reflexionen entstehen an Materialgrenzen, wenn Unterschiede in Permittivität, Permeabilität oder Leitfähigkeit vorliegen. Die Stärke und Form der reflektierten Signale liefern Hinweise auf die Eigenschaften und die Geometrie der jeweiligen Grenzflächen. \\
Eine häufig verwendete Messanordnung ist die Zero-Offset-Methode, bei der Sender und Empfänger sich in unmittelbarer Nähe befinden und gemeinsam entlang eines Profils bewegt werden. Die Laufzeit der Signale zwischen Aussendung und Empfang sowie die Position der Antennen ermöglichen Rückschlüsse auf die Tiefe und Lage der reflektierenden Strukturen. Die Messergebnisse werden als Radargramme dargestellt: Die horizontale Achse zeigt die Position entlang des Messprofils, die vertikale Achse die Laufzeit oder Tiefe der reflektierten Signale. Als Messposition wird üblicherweise der Mittelpunkt zwischen Sende- und Empfängerantenne gewählt. Die Intensität der Bildpunkte entspricht der Amplitude der empfangenen Reflexionen (\cite{annan_ground_2003}). \\
Radargramme liefern detaillierte Informationen über die innere Struktur des Bodens. Sie ermöglichen die Erkennung von Schichtgrenzen, Untergrundstrukturen, Leitungen oder archäologischen Objekten. Die Interpretation der Radargramme ist jedoch anspruchsvoll, da verschiedene Faktoren wie Signalabschwächung, Mehrfachreflexionen oder Störsignale die Darstellung beeinflussen können. Die Auswertung erfolgt meist manuell durch erfahrene Fachleute, die relevante Muster identifizieren, Störsignale erkennen und geophysikalisch bedeutungsvolle Strukturen voneinander abgrenzen. \\
Der manuelle Charakter dieser Arbeitsschritte bringt jedoch mehrere Probleme mit sich: Die Auswertung ist zeitintensiv und erfordert ein hohes Maß an Fachwissen und Erfahrung. Insbesondere bei großen oder komplexen Datensätzen stößt die manuelle Analyse schnell an ihre Grenzen. Die steigende Verfügbarkeit hochauflösender und automatischer Messsysteme, wie beispielsweise Multichannel-GPR, führt zudem zu immer größeren Datenmengen, deren effiziente und objektive Auswertung zunehmend zur Herausforderung wird.

\subsection{Bezug zur Seismologie}

Die Seismologie ist genau wie das Ground Penetrating Radar ein geophysikalisches Verfahren zur Untersuchung des Untergrunds. Die Verfahren haben eine sehr ähnliche methodische Basis und basieren auf ähnlichen physikalischen Prinzipien \parencite{forte_review_2017}. Während GPR elektromagnetische Wellen nutzt, um den Untergrund zu durchdringen, basiert die Seismologie auf der Analyse von mechanischen, speziell seismischen Wellen, die durch Erdbeben oder künstliche Explosionen erzeugt werden \parencite{carcione_acoustic-electromagnetic_1995}. \textcite{forte_review_2017} führen noch weiter aus, dass aufgrund der Ähnlichkeiten zwischen seismischer und hochfrequent-elektromagnetischer Wellenausbreitung, seismische Verarbeitungstechniken auf Radardaten angewendet werden können. \\
In der Seismologie zeigte sich schon früh, dass die Auswertung großer Datenmengen eine erhebliche Herausforderung darstellt (\cite{dowla_seismic_1990}). Nach dem Ende des Kalten Krieges wurden internationale Verträge geschlossen, die Nukleartests weltweit untersagten und durch ein globales seismisches Netzwerk überwacht werden sollten. Dabei mussten künstliche Explosionen von natürlichen Erdbeben zuverlässig unterschieden werden \parencite{dahlmann_monitoring_1977}. Die erfassten Datenmengen waren so groß, dass eine manuelle Analyse nicht mehr praktikabel war. \\
Um diese Herausforderungen zu bewältigen, arbeitete \textcite{dowla_seismic_1990} gezielt an automatisierten Ansätzen zur Auswertung seismischer Daten. Aufbauend auf diesen Arbeiten entwickelte er gemeinsam mit \textcite{maurer_seismic_1992} die Anwendung von Self-Organizing Maps (SOMs), zu deutsch auch Selbst-Organisierende Karten. Mit SOMs konnten komplexe Strukturen in seismischen Daten segmentiert und interpretiert werden, wobei die Ähnlichkeitsbeziehungen der Daten erhalten blieben. Diese Methode liefert objektive, reproduzierbare und visuell nachvollziehbare Ergebnisse und eignet sich daher besonders für die geophysikalische Datenanalyse.

\subsection{Studienziele}

Im Rahmen meiner Arbeit übertrage ich Methoden des maschinellen Lernens aus der Seismologie auf GPR-Daten, um die Segmentierung von Radargrammen zu automatisieren. Dabei möchte ich folgende Forschungsfragen beantworten:

\begin{enumerate}
    \item Können die seismischen Feature - Berechnungen auf Ground Penetrating Radar-Daten angewendet werden und erzeugen sinnvolle Ergebnisse?
    \item Kann die Self-Organizing Map aus den berechneten Features sinnvolle und aussagekräftige Cluster bilden?
    \item Lässt sich ein trainiertes SOM-Modell auf andere Radargramme übertragen?
\end{enumerate}
Ausgehend von den oben formulierten Forschungsfragen gliedert sich meine Arbeit wie folgt: Zunächst beschreibe ich im Methodenteil, wie ich die Radargramme vorprozessiert habe, um eine optimale Ausgangsbasis für die weitere Analyse zu schaffen. Dazu zählen Schritte wie die Normalisierung der Daten, die Entfernung von Störsignalen und die Auswahl relevanter Ausschnitte. Anschließend erläutere ich alle berechneten Features im Detail, darunter seismische Attribute wie Amplitude, Frequenz und Texturmerkmale, die als Eingabe für die Segmentierung dienen. Im nächsten Abschnitt beschreibe ich die Optimierung der Self-Organizing Map (SOM), wobei ich Parameter wie die Kartenstruktur, Lernrate und Anzahl der Iterationen systematisch angepasst habe, um eine möglichst sinnvolle Clusterbildung zu erreichen. Abschließend diskutiere ich die Ergebnisse und bewerte sie hinsichtlich ihrer Aussagekraft, Übertragbarkeit und praktischen Relevanz. \\
Um die inhaltliche Kohärenz und den vorgegebenen Umfang der Bachelorarbeit zu wahren, wird auf eine weiterführende Bestimmung der Untergrundmaterialien oder GPR-Fazies gemäß \textcite{corradini_day_2023} verzichtet.