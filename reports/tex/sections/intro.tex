\section{Einführung}

Das Ground Penetrating Radar (Bodenradar) ist ein geophysikalisches Messverfahren, das elektromagnetische Wellen im Hochfrequenzbereich nutzt, um Reflexionen an Materialgrenzen im Untergrund zu erfassen. Aus diesen Signalen lassen sich Informationen über die innere Struktur des Bodens gewinnen – etwa zur Erkennung von Schichtgrenzen, Hohlräumen, Störungen oder archäologischen Objekten. Aufgrund seiner hohen Auflösung und zerstörungsfreien Arbeitsweise findet das Bodenradar breite Anwendung in Bereichen wie Geologie, Archäologie, Bauwerksdiagnostik und Umwelttechnik. Die Messergebnisse werden in Form sogenannter Radargramme dargestellt – bildhafte Datenprofile, die die reflektierten Signale entlang eines Messprofils über die Tiefe hinweg abbilden.\\
Die Auswertung von Radargrammen erfolgt derzeit meist manuell durch erfahrene Fachleute. Mithilfe geschulter visueller Interpretation identifizieren sie relevante Muster, erkennen und bewerten Störsignale und grenzen geophysikalisch bedeutungsvolle Strukturen voneinander ab. Diese Segmentierung dient unter anderem der Klassifizierung geologischer Einheiten (Fazies), der Lokalisierung von Reservoirgrenzen oder der Kartierung von Anomalien. Der manuelle Charakter dieser Arbeitsschritte bringt jedoch mehrere Probleme mit sich: Die Auswertung ist zeitintensiv, nur begrenzt reproduzierbar und stößt bei großen oder komplexen Datensätzen schnell an ihre Grenzen. \\
Ein vielversprechender Weg zur Bewältigung dieser Herausforderungen ist der Einsatz von Methoden des maschinellen Lernens. Ziel ist es, Muster in Daten automatisiert zu erkennen und zu gruppieren, um strukturierte, nachvollziehbare und skalierbare Auswertungen zu ermöglichen. Damit Algorithmen solche Muster erkennen können, müssen die Daten zunächst in eine geeignete, „lesbare“ Form gebracht werden. Eine wichtige Rolle spielt dabei das sogenannte Clustering, also das automatische Gruppieren von Datenpunkten mit ähnlichen Eigenschaften – ohne dass vorher bekannte Klassen vorhanden sein müssen. \\
Eine besonders leistungsfähige Clustering-Methode ist die Self-Organizing Map (Selbst-organisierende Karte), die in den 1980er-Jahren vom finnischen Wissenschaftler Teuvo Kohonen entwickelt wurde. Dabei handelt es sich um ein künstliches neuronales Netz, das die Aufgabe hat, hochdimensionale Daten – etwa Zeitreihen, Signalformen oder abgeleitete Attribute – auf eine niedrigdimensionale, meist zweidimensionale Gitterstruktur abzubilden. Das Besondere an dieser Methode ist, dass sie bei der Abbildung die Nachbarschaftsbeziehungen der Daten bewahrt: Ähnliche Datenpunkte werden in benachbarten Bereichen der Karte dargestellt, während unähnliche weit voneinander entfernt erscheinen. Das Netz lernt durch einen iterativen Anpassungsprozess, in dem jeder Eingabewert einem passenden Neuron (dem sogenannten „Best Matching Unit“) zugeordnet wird, das zusammen mit seinen Nachbarn schrittweise an die Eingabedaten angepasst wird. Dadurch entsteht eine geordnete, interpretierbare Kartenstruktur. \\
In der Seismik, einem eng verwandten Feld der geophysikalischen Erkundung, hat sich der Einsatz von Self-Organizing Maps in den letzten Jahren als äußerst nützlich erwiesen. Dort dienen sie beispielsweise der automatisierten Klassifikation von Reflexionssignalen, der Erkennung diskontinuierlicher Strukturen oder der Identifikation geologischer Fazies auf Grundlage seismischer Attribute. Ihre Fähigkeit, große Datenmengen strukturiert zu gruppieren und gleichzeitig interpretierbare Karten zu erzeugen, macht sie zu einem wertvollen Werkzeug in der geophysikalischen Datenanalyse. \\
Ziel dieser Arbeit ist es, zu untersuchen, ob sich diese in der Seismik etablierten Methoden auf Radargramme übertragen lassen. Konkret geht es darum, seismische Attributverfahren auf Bodenradardaten anzuwenden und diese anschließend mithilfe einer Self-Organizing Map zu clustern. Dadurch soll eine automatisierte Segmentierung von Radargrammen ermöglicht werden, die effizient, objektiv und visuell nachvollziehbar ist. \\
Im Zentrum der Untersuchung steht die Frage: Können Self-Organizing Maps, wie sie in der seismischen Datenverarbeitung verwendet werden, erfolgreich zur automatisierten Segmentierung von Radargrammen eingesetzt werden? Dazu wird ein angepasster Analyse-Workflow entwickelt, der geeignete Merkmale aus den Radargrammen extrahiert, diese vorbereitet und anschließend mit einer Self-Organizing Map clustert. Die entstehenden Cluster werden hinsichtlich ihrer geophysikalischen Aussagekraft bewertet und, wenn möglich, mit vorhandenen manuellen Interpretationen verglichen. \\
Nicht Gegenstand dieser Arbeit ist die physikalische Modellierung der Radarwellenausbreitung oder die Entwicklung neuer Lernalgorithmen. Im Mittelpunkt steht die Anwendung und Bewertung eines etablierten maschinellen Lernverfahrens auf ein neues, bisher kaum automatisiert analysiertes Datenformat. Ziel ist es, die Potenziale und Grenzen dieses Ansatzes herauszuarbeiten und eine Grundlage für zukünftige Anwendungen in der geophysikalischen Praxis zu schaffen. \\